% simple.tex

\documentclass{article}[12pt,a4paper]

\usepackage{hyperref}
\hypersetup{
    colorlinks,
    citecolor=black,
    filecolor=black,
    linkcolor=black,
    urlcolor=black
}

\usepackage{tikz}
\usetikzlibrary{shapes.geometric, arrows}

\tikzstyle{startstop} = [rectangle, rounded corners, minimum width=3cm, minimum height=1cm,text centered, draw=black, fill=red!30]
\tikzstyle{io} = [trapezium, trapezium left angle=70, trapezium right angle=110, minimum width=3cm, minimum height=1cm, text centered, draw=black, fill=blue!30]
\tikzstyle{process} = [rectangle, minimum width=3cm, minimum height=1cm, text centered, text width=3cm, draw=black, fill=orange!30]
\tikzstyle{decision} = [diamond, minimum width=3cm, minimum height=1cm, text centered, draw=black, fill=green!30]
\tikzstyle{arrow} = [thick,->,>=stealth]

\usepackage{listings}
\usepackage{color}

\definecolor{codegreen}{rgb}{0,0.6,0}
\definecolor{codegray}{rgb}{0.5,0.5,0.5}
\definecolor{codepurple}{rgb}{0.58,0,0.82}
\definecolor{backcolour}{rgb}{0.95,0.95,0.92}
 
\lstdefinestyle{python}{
    backgroundcolor=\color{backcolour},   
    commentstyle=\color{codegreen},
    keywordstyle=\color{magenta},
    numberstyle=\tiny\color{codegray},
    stringstyle=\color{codepurple},
    basicstyle=\footnotesize,
    breakatwhitespace=false,         
    breaklines=true,                 
    captionpos=b,                    
    keepspaces=true,                 
    numbers=left,                    
    numbersep=5pt,                  
    showspaces=false,                
    showstringspaces=false,
    showtabs=false,                  
    tabsize=2
}

\lstset{style=python}

\begin{document}

\title{WJEC GCE Computing CG2 - Extended Task}

\author{Candidate Name: Daniel Roberts\\
        Candidate Number: 4699\\
        Centre Name: Shrewsbury Sixth Form College\\
        Centre Number: 29285}

\date{}

\maketitle

\tableofcontents

\cleardoublepage

\part{Analysis and Design}
This part of the documentation contains the analysis that was performed on Parkwood Vale Harriers, taking into account what the running club asked for in their brief, and exploring these requirements. It also covers the preliminary design that was created for the system, including the interface design for every page, the design of the data structures and process design, detailing the different algorithms that have been used, and how the system interacts with itself.

\section{Problem Definition}
\subsection{Background}
Parkwood Vale Harriers is a running club that serves the fitness needs of many different members, through regular training sessions, as well as races. The club gets involved in the local community, a position that consists, in part, of raising money for local charities. 

Recently, the club has decided to raise money for one of the charities by putting on a relay event, wherein a team of runners will run, non-stop, from John O\textsc{\char13} Groats to Land’s End, in the shortest time possible. The team will consist of eight members, and each runner will run for an hour at a time, whilst the others rest in the minibus. The entire trip is estimated to take three days and as a result of this, each member of the team will have to be very fit.

In order to increase their chances of completing the run, the club has decided to find out the most appropriate team, based on the results of a physically challenging training programme. This programme will consist of running, cycling and swimming, and will serve to ensure that only the top members of the club are included in the team.

\subsection{Broad Aims}
The running club has commissioned  a computer based system that will allow the runners to keep an accurate record of their running, cycling and swimming sessions. This data will then be used to calculate an informed decision of the most appropriate team for the relay race.

The system must allow each runner to monitor their progress during the training programme, clearly showing them the extent to which they have improved. As such, the system must provide an interface to allow the runner to add each training session they perform, with spaces for the type of training, the time spent, how hard they pushed themselves, and other such parameters. Using this data, the system must then calculate the number of calories burned in the training session, providing a series of data points through which the performance of the runner can be monitored.

To further aid in this, the system must be able to output these training sessions in a clear format that the runner is able to clearly understand. This can be achieved through the use of tables to display each training session in a listed, tabular format, as well as through graphs and charts to display the data in a graphical form; this maes overall performance trends easy to visualise.

Due to the nature of the system, the ability to store certain personal information, such as the name, age and weight of the runner, must also be included. The runner should have the ability to input this information themselves, most likely upon first use of the system. There should be the ability to modify this data, in the result of an error being made or the circumstances of the runner changing.

A key aspect of the system, and one that is key to promoting the competitive values of the club, is the ability to compare results with other participants in the program. This area of the system should allow runners to compare key aspects of their performance, such as the results of their individual training sessions, as well as their overall performance over time in all three of the training activities.

As the main point of the system, the ability to select the final team must also be included. By analysing the data points provided by the runners, the system should be able to choose the most appropriate team.

\subsection{Limitations}
Though the brief provided by the running club contains several good ideas and acts as an effective base upon which to work, there are a number of areas which the running club has not thought about that could be factored into the solution, creating a more effective system. 

One very important factor that the running club has left out is security. In a system like this, where intensely personal data is being stored, including data that the user may not which to become public, such as their weight, it is important that the data is stored in a secure manner that allows only those with the correct permissions to access it. 

Another issue with the brief is that of an objective decision being made when selecting the team. Running a marathon is about far more than just physical fitness; more personal aspects, such as how well the runners get along and different roles within the team, should also be taken into account for maximum efficiency. The system would be unable to do this (without each runner giving their opinion on the others, which is unrealistic), and so the team it comes up with may not be the most appropriate choice. 

Another limitation in the system is that data will have to be entered manually: there is no way of taking the data from some sort of personal tracking device. This could result in some issues with accuracy, or even with malpractice: people entering exaggerated data in order to manipulate the rankings and make themselves seem better. A mixture of validation and verification can be put in place to prevent this, such as ensuring users cannot go for a straight eight hour swim (something which is obviously unrealistic), but this will be unable to catch all cases of exaggeration; it is therefore necessary to rely on the goodwill and sportsmanship of the runners.

Furthermore, the system relies on the premise that the runners will add every training session they perform to the application. It is not unlikely that they will go on unsolicited training sessions that they do not bother adding, or they may simply forget. There is no foolproof manner to prevent these occurences, but a number of steps can be taken to reduce their likelihood, such as by making the process of adding a session as simple as possible - the easier the process is, the more likely the runner is to do it.

In addition, the brief asks for only the top eight members of the running team to be calculated. This does not take into account the possibilities of injuries or runners dropping out for other reasons; as such, 

\subsection{Assumptions}
Throughout the system, a number of assumptions have been made in order to increase the ease of development. 

One of these is that in each individual training session, only one method of exercise will be used, such as breaststroke for an entire swimming session or a leisurely speed for an entire cycling session. Though this is alleviated to some extent by the ability to add multiple sessions for each sport on a single day, the assumption still has to be made. 

In addition to this, the assumption that each session lasts for at least an hour has been made: the time picker only uses stages of sixty minutes, as opposed to thirty or fifteen.

Naturally, the system also assumes that the user is relatively proficient with a computer based interface. Effort has been put in to make the system as user friendly and as easy to use as possible, but someone using a computer for the first time will undoubtedly find it more difficult than someone with at least a little experience.

\subsection{Objectives}
In order to create the system to an acceptable quality, a number of objectives will have to be fulfilled. The system must:

\begin{itemize}
    \item Have a simple, clear interface that allows tasks to be performed easily.
    \item Allow the runner to add, view, update and, if they choose, delete their personal information, such as their name, email address, date of birth and phone number.
    \item Allow the runner to add, view and delete the training sessions they perform in over the course of the training period; this will include information like the date and time of the session, the speed they were training at, and how well it went.
    \item Persistently store this data in appropriately named tables in a database.
    \item Ensure the security of this data by giving each runner their own personal account, protected by a username and an encrypted password.
    \item Calculate the number of calories burned in each training session, by taking into account the runner's weight, the time spent on the session, the nature of the session, and how well the runner thought it went.
    \item Allow the user to view graphical, interactive graphs of their training sessions, allowing them to easily view trends in their performance.
\end{itemize}

\subsection{Justification of Proposed Solution}
When building a solution to a problem like the one faced by Parkwood Vale Harriers, there are generally two methods available: utilising the features of an existing software package, such as Microsoft Office Access, or programming an existing solution in a programming language, such as Visual Basic or Python. Both have their advantages and drawbacks: by utilising an existing package, much of the system will already be developed; it only remains to manipulate the system to meet the needs of the brief; but, on the other hand, one can be limited by the restrictions of the software package, perhaps preventing the final solution being as capable as it might otherwise have been.

An original solution created using a programming language would suffer from rather the opposite issues: as a result of the practically endless results that can be achieved through their use, there is a definite learning curve that is not present (or is less exacerbated) in software packages; as a result of this, development time will likely be considerably longer. Despite these drawbacks, it is clear that, if a programming language is used, the final solution is likely to be of a higher quality: not only can more advanced features be implemented, these features - as well as those of a more basic level - are likely to be of a higher quality.  In addition, the developer will have a greater understanding of the system, as they will have built it entirely themselves (aside from any additional packages/libraries used); this will aid in areas like debugging, and will also make it easier to write up system documentation and the like.

The question then falls to exactly which programming language is the most appropriate. There are a large number of languages available, ranging from \textit{compiled} languages like Java, C\# and Visual Basic to \textit{interpreted} languages like Ruby, Python and PHP. The differences between compiled and iterpreted languages are complex and varied, but, in essence, compiled languages are likely to perform algorithms more quickly (due to directly using the native code of the target machine), whereas code written in an interpreted language can be executed "on the fly", so to speak, increasing development speed. 

Another factor to consider is the \textit{typing system} used by the language. Two types of typing exist: \textit{static} and \textit{dynamic}; in general, compiled languages use static typing, whilst interpreted languages use dynamic typing. In a statically typed language, the data type of a variable must be defined. Consider the following Python code:

\begin{lstlisting}[language=Python]
array_of_numbers = [10, 20, 30]
\end{lstlisting}
Compare to the equivalent of this code in Java:

\begin{lstlisting}[language=Java]
int[] arrayOfNumbers = {10, 20, 30}
\end{lstlisting}

Though Java undoubtedly has many admirable qualities, it is not a language commonly noted for its ease of use or the beauty of its syntax. Though these issues may appear superficial,

\section{Data Structures and Methods of Access}
In order to persistently store the runner's data, a database is needed. As is the custom with applications of this sort, there will be one single database file, within which will be a number of tables. The system will also make use of a number of arrays and JSON structures, to temporarily store data.

\subsection{Database Tables}
The system uses the SQLite database system.
\\\\\textbf{\textit{A note on validation: }}\textit{SQLite does not perform any validation itself. All validation is performed during the processing of the data, before it is added into the database. As such, details on the validation performed on the data saved to these tables can be found in their relevant section.}


\subsubsection{Users Table}
This table stores the personal information for each runner. Whenever a runner creates an account, the data they input into the registration form will end up in this table.

\begin{table}[h]
\begin{tabular}{|l|l|l|l|l|}
\hline
Field Name     & Primary Key & Typical Data         & Data Type \\ \hline
id             & True        & 01                   & Integer   \\ \hline
name           & n/a         & John Smith           & String    \\ \hline
email          & n/a         & john@smith.com       & String    \\ \hline
username       & n/a         & john5                & String    \\ \hline
password\_hash & n/a         & pbkdf2:sha1:1000\$02 & String    \\ \hline
dob            & n/a         & 1997-02-02           & Date      \\ \hline
phone          & n/a         & 07722895880          & String    \\ \hline
weight         & n/a         & 74                   & Integer   \\ \hline
distance       & n/a         & less than 1          & String    \\ \hline
joined         & n/a         & 2015-01-04           & Date      \\ \hline
charity\_event & n/a         & True                 & Boolean   \\ \hline
\end{tabular}
\caption{Users Table}
\end{table}

\subsubsection{Activities Table}

\begin{table}[h]
\begin{tabular}{|l|l|l|l|}
\hline
\textbf{Field Name} & \textbf{PK / FK} & \textbf{Typical Data} & \textbf{Data Type} \\ \hline
id                  & Primary          & 01                    & Integer            \\ \hline
sport               & n/a              & running               & String             \\ \hline
effigy              & n/a              & 5 mph                 & String             \\ \hline
date                & n/a              & 2015-01-04            & Date               \\ \hline
start               & n/a              & 8:00AM                & String             \\ \hline
finish              & n/a              & 10:00AM               & String             \\ \hline
hours               & n/a              & 2                     & Integer            \\ \hline
opinion             & n/a              & Brilliant             & String             \\ \hline
thoughts            & n/a              & It was great.         & String             \\ \hline
user\_id            & Foreign          & 02                    & Integer            \\ \hline
\end{tabular}
\caption{Activities Table}
\end{table}

\section{User Interface Design}

\cleardoublepage


\part{Program Documentation}


\cleardoublepage


\part{Testing and Evaluation}

\addcontentsline{toc}{section}{Unnumbered Section}

\end{document}